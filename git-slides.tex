\documentclass{beamer}
\usepackage[utf8]{inputenc}
\usepackage[english,russian]{babel}
\usepackage{hyperref}
\usepackage{xcolor}
\usepackage{graphicx}

\usetheme{Boadilla}
\usecolortheme{seahorse}
\setbeamercovered{transparent}% Allow for shaded (transparent) covered items

\AtBeginSection[]
{
  \begin{frame}
    \frametitle{Содержание}
    \tableofcontents[currentsection]
  \end{frame}
}

\begin{document}

\title[]{Git: распределённая система контроля версий}
\author{Н.\,Д.~Кудасов}
\institute{МГУ им. Ломоносова}
\date{Москва, 2013}

\begin{frame}
\addtocounter{framenumber}{-1}
\maketitle
\end{frame}

\section{Введение}

\begin{frame}
  \frametitle{Системы контроля версий}

  Системы контроля версий позволяют отслеживать изменения в коде, документах и пр.

  \begin{block}{Централизованные системы}
  Единый сервер для хранения документов и истории.
  Централизованный контроль над хранилищем.
  \end{block}

  \begin{block}{Распределенные системы}
  Каждый пользователь имеет собственную версию репозитория.
  Изменение могут передаваться между репозиториями.
  \end{block}
\end{frame}

\begin{frame}
  \frametitle{Зачем нужны системы контроля версий?}
  Случалось ли, что Вы
  \begin{itemize}
    \item изменили код, обнаружили ошибку и захотели откатиться?
    \item потеряли код, а имеющийся бекап оказался устаревшим?
    \item должны были поддерживать несколько версий программы?
    \item хотели посмотреть, чем отличаются две версии программы?
    \item хотели обнаружить, какое изменение внесло ошибку в программу?
    \item хотели просмотреть историю исходного кода?
    \item хотели отправить изменение для чужого кода?
    \item хотели выложить код, чтобы другие могли работать над ним?
    \item хотели посмотреть, сколько работы проделано? Когда? Кем?
    \item хотели попробовать что-то новое, не ломая основного кода?
    \item и т.д.
  \end{itemize}
\end{frame}

\begin{frame}
  \frametitle{Системы контроля версий: примеры}
  \begin{block}{Централизованные системы}
    \begin{itemize}
      \item CVS
      \item Subversion (SVN)
      \item Visual SourceSafe, Vault и т.д.
    \end{itemize}
  \end{block}

  \begin{block}{Распределенные системы}
    \begin{itemize}
      \item Git,
      \item Darcs,
      \item Mercurial, Bazaar,
      \item Bitkeeper и т.д.
    \end{itemize}
  \end{block}
\end{frame}

\section{Git: модель репозитория}

\section{Git: основные операции}

\section{GitHub: открытое взаимодействие}

\end{document}

