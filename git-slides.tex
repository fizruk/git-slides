\documentclass{beamer}
\usepackage[utf8]{inputenc}
\usepackage[english,russian]{babel}
\usepackage{hyperref}
\usepackage{xcolor}
\usepackage{graphicx}

\usetheme{Boadilla}
\usecolortheme{seahorse}
\setbeamercovered{transparent}% Allow for shaded (transparent) covered items

\AtBeginSection[]
{
  \begin{frame}
    \frametitle{Содержание}
    \tableofcontents[currentsection]
  \end{frame}
}

\begin{document}

\title[]{Git: распределённая система контроля версий}
\author{Н.\,Д.~Кудасов}
\institute{МГУ им. Ломоносова}
\date{Москва, 2013}

\begin{frame}
\addtocounter{framenumber}{-1}
\maketitle
\end{frame}

\section{Введение}

\begin{frame}
  \frametitle{Системы контроля версий}

  Системы контроля версий позволяют отслеживать изменения в коде, документах и пр.

  \begin{block}{Централизованные системы}
  Единый сервер для хранения документов и истории.
  Централизованный контроль над хранилищем.
  \end{block}

  \begin{block}{Распределенные системы}
  Каждый пользователь имеет собственную версию репозитория.
  Изменение могут передаваться между репозиториями.
  \end{block}
\end{frame}

\begin{frame}
  \frametitle{Зачем нужны системы контроля версий?}
  Случалось ли, что Вы
  \begin{itemize}
    \item изменили код, обнаружили ошибку и захотели откатиться?
    \item потеряли код, а имеющийся бекап оказался устаревшим?
    \item должны были поддерживать несколько версий программы?
    \item хотели посмотреть, чем отличаются две версии программы?
    \item хотели обнаружить, какое изменение внесло ошибку в программу?
    \item хотели просмотреть историю исходного кода?
    \item хотели отправить изменение для чужого кода?
    \item хотели выложить код, чтобы другие могли работать над ним?
    \item хотели посмотреть, сколько работы проделано? Когда? Кем?
    \item хотели попробовать что-то новое, не ломая основного кода?
    \item и т.д.
  \end{itemize}
\end{frame}

\begin{frame}
  \frametitle{Системы контроля версий: примеры}
  \begin{block}{Централизованные системы}
    \begin{itemize}
      \item CVS
      \item Subversion (SVN)
      \item Visual SourceSafe, Vault и т.д.
    \end{itemize}
  \end{block}

  \begin{block}{Распределенные системы}
    \begin{itemize}
      \item Git,
      \item Darcs,
      \item Mercurial, Bazaar,
      \item Bitkeeper и т.д.
    \end{itemize}
  \end{block}
\end{frame}

\begin{frame}
  \frametitle{Git}
  
  \includegraphics[width=2cm]{images/git-logo.png}

  Одна из наиболее популярных систем контроля версий.

  \begin{block}{Основные преимущества}
    \begin{itemize}
      \item пример хорошо написанной и производительной программы на C;
      \item компактный репозиторий (эффективное сжатие и хранение истории);
      \item множество копий/бекапов;
      \item произвольная организация работы с несколькими репозиториями;
      \item промежуточная область для коммитов;
      \item простые в использовании ветки.
    \end{itemize}
  \end{block}
\end{frame}

\begin{frame}
  \frametitle{Git: проекты}
  Git подходит для использования в любом проекте (как и любая
  распределенная система контроля версий).

  \begin{block}{Примеры использований}
    \begin{itemize}
      \item ядро ОС Linux и смежные проекты;
      \item gcc и смежные проекты;
      \item ОС Android,
      \item Qt, Cairo,
      \item Ruby on Rails, jQuery,
      \item GitHub.com,
      \item огромное количество других проектов.
    \end{itemize}
  \end{block}
\end{frame}

\section{Git: модель репозитория}

\begin{frame}
  \frametitle{История: хранение изменений}
  Большинство систем контроля версий хранят историю в виде списка
  изменений для каждого файла. В таких системах информация о проекте
  представляется в виде множества файлов и изменений для каждого файла:

  \begin{figure}
     \includegraphics[width=10cm]{images/history-deltas.png}
  \end{figure}
\end{frame}

\begin{frame}
  \frametitle{История: хранение снимков}
  Git смотрит на дело по-другому: он хранит историю как ряд снимков для
  миниатюрной файловой системы. При каждом коммите создаётся очередной снимок
  проекта. Если файл не менялся, в новом снимке будет ссылка на файл из предыдущего
  снимка.

  \begin{figure}
     \includegraphics[width=10cm]{images/history-snapshots.png}
  \end{figure}
\end{frame}

\begin{frame}
  \frametitle{Три состояния}
  Файлы (документы) в Git могут находиться в одном из трёх состояний: зафиксированном,
  изменённом и подготовленном. {\it Зафиксирован} означает, что файл сохранён в истории.
  Подготовленные файлы --- это изменённые файлы, отмеченные для включения в следующий
  коммит. Таким образом, в Git существуют три части:

  \begin{figure}
     \includegraphics[width=6.4cm]{images/git-sections.png}
  \end{figure}
\end{frame}

\begin{frame}
  \frametitle{Основы}
  \begin{block}{Коммит}
    \begin{itemize}
      \item набор файлов (снимок);
      \item ссылки на родительские коммиты;
      \item уникальный SHA1-код (хеш).
    \end{itemize}
  \end{block}

  \begin{block}{Ссылка}
    \begin{itemize}
      \item просто ссылка на коммит;
      \item имеют имена, по умолчанию создается ссылка \texttt{master};
      \item ссылка, активная на текущий момент, имеет синоним \texttt{HEAD}.
    \end{itemize}
  \end{block}

  \begin{figure}
     \includegraphics[width=5cm]{images/git-simple-repo.png}
  \end{figure}
\end{frame}

\begin{frame}
  \frametitle{Первоначальная настройка}
  Прежде, чем начать работать с Git, ему необходимо сообщить немного о себе:

  \begin{figure}
     \includegraphics[width=12cm]{images/git-config.png}
  \end{figure}

  Эта информация будет использована в истории, чтобы можно было проследить,
  кто отвественен за какие изменения.

  Опция \texttt{--global} вносит изменения в файл \texttt{.gitconfig} в
  домашней директории. Этот файл используется по умолчанию. Если опция
  не указана, настройки применяются только к текущему репозиторию.
\end{frame}

\begin{frame}
  \frametitle{Создание репозитория}
  При создании репозитория, Git создаёт служебную директорию \texttt{.git/}
  в корне проекта:
  \begin{figure}
     \includegraphics[width=10cm]{images/git-init.png}
  \end{figure}

  Команда \texttt{git add} регистрирует изменения, которые попадут в очередной коммит.
  Команда \texttt{git commit} создаёт коммит.

  \begin{figure}
     \includegraphics[width=10cm]{images/git-add-commit.png}
  \end{figure}

  Этих команд уже хватает для работы над проектом с Git!
\end{frame}

\begin{frame}
  \frametitle{Просмотр истории}
  Команда \texttt{git log} позволяет показать историю изменений:

  \begin{figure}
     \includegraphics[width=10cm]{images/git-log.png}
  \end{figure}

  \begin{figure}
     \includegraphics[width=10cm]{images/git-log-oneline-decorate.png}
  \end{figure}
\end{frame}

\begin{frame}
  \frametitle{Состояние проекта}
  Команда \texttt{git status} позволяет посмотреть список изменений
  (зарегистрированных, незарегистрированных, новых файлов, и т.д.):

  \begin{figure}
    \includegraphics[width=10cm]{images/git-status.png}
  \end{figure}
\end{frame}

\begin{frame}
  \frametitle{Просмотр изменений}
  Команда \texttt{git diff} позволяет посмотреть сами изменения:

  \begin{figure}
    \includegraphics[width=10cm]{images/git-diff.png}
  \end{figure}
\end{frame}

\section{Git: ветвление}

\begin{frame}
  \frametitle{Создание ветки}
  Ветки в системах контроля версий позволяют отклониться от основной
  линии разработки. В Git (в отличие от многих други систем) ветки
  легковесны, что позволяет работать одновременно с тысячами веток
  в больших проектах.
  Команда \texttt{git branch} создаёт новую ветку, а \texttt{git checkout}
  меняет текущую ветку:

  \begin{figure}
    \includegraphics[width=10cm]{images/git-branch.png}
  \end{figure}
\end{frame}

\begin{frame}
  \frametitle{Что происходит?}
  \begin{exampleblock}{}
    Ветки --- это просто ссылки!
  \end{exampleblock}

  \begin{columns}[c]
  \column{.4\textwidth}
    \only<1>{Исходный репозиторий}\only<2>{\texttt{git branch test}}\only<3>{\texttt{git checkout test}}\only<4>{\texttt{git commit}}\only<5>{\texttt{git checkout master}}\only<6>{\texttt{git commit}}
  \column{.6\textwidth}
    \begin{figure}
      \includegraphics[width=7cm]<1>{images/git-simple-repo.png}
      \includegraphics[width=7cm]<2>{images/git-simple-repo2.png}
      \includegraphics[width=7cm]<3>{images/git-simple-repo3.png}
      \includegraphics[width=7cm]<4>{images/git-simple-repo4.png}
      \includegraphics[width=7cm]<5>{images/git-simple-repo5.png}
      \includegraphics[width=7cm]<6>{images/git-simple-repo6.png}
    \end{figure}
  \end{columns}
\end{frame}

\begin{frame}
  \frametitle{Слияние веток}
  При слиянии веток Git находит наилучшего общего предка двух коммитов,
  на которые ссылаются ветки. Если в ветке, в которую происходит слияние,
  нет новых коммитов, то указатель просто передвигается. Иначе создаётся
  новый коммит {\it с двумя предками}:

  \begin{figure}
    \includegraphics[width=10cm]<1>{images/git-simple-repo-merge.png}
    \includegraphics[width=10cm]<2>{images/git-simple-repo-merge2.png}
  \end{figure}

  \begin{center}
  \only<1>{Исходный репозиторий}
  \only<2>{\texttt{git merge test}}
  \end{center}
\end{frame}

\section{Git: удалённые репозитории}

\begin{frame}
  \frametitle{Клонирование репозитория}
  Команда \texttt{git clone} полностью копирует репозиторий.

  \begin{figure}
    \includegraphics[width=10cm]{images/git-clone.png}
  \end{figure}

  Команда \texttt{git remote} перечисляет список удалённых репозиториев:

  \begin{figure}
    \includegraphics[width=10cm]{images/git-remote.png}
  \end{figure}
\end{frame}

\section{GitHub: открытое взаимодействие}

\end{document}

